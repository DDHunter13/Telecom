\include{settings}

\setcounter{tocdepth}{3}

\begin{document}	% начало документа

% Титульная страница
\include{titlepage}

% Содержание
\include{ToC}


\section{Цель работы}
Изучение частотной и фазовой модуляции и демодуляции сигналов.
\section{Постановка задачи}
Сгенерировать однотональный сигнал низкой частоты, выполнить фазовую и частотную модуляцию и демодуляцию. Посмотреть, как модуляция влияет на спектр сигнала.

\section{Теоретическая информация}

\subsection{Модуляция}
Перенос спектра сигналов из низкочастотной области на заданную частоту, т.е. в выделенную для их передачи область высоких частот выполняется операцией модуляции.   

В канале связи для передачи данного сигнала выделяется определенный диапазон высоких частот и формируется вспомогательный периодический высокочастотный сигнал $u(t)   =   f(t;   a_1,   a_2,   ...   a_m)$. Совокупность параметров $a_i$ определяет форму вспомогательного сигнала. Если сделать значения сигнала $s(t)$ зависимыми пропорционально зависимым от значения $s(t)$ во времени (или по любой другой независимой переменной), то форма сигнала $u(t)$ приобретает новое свойство. Она служит для переноса информации, содержащейся в сигнале $s(t)$. Сигнал $u(t)$ называется несущей,  а физический процесс переноса информации на
параметры несущего сигнала     –     его модуляцией. 

Исходный информационный сигнал $s(t)$ называют модулирующим. Обратную операцию выделения модулирующего сигнала из модулированного колебания называют демодуляцией.

\subsection{Генерация однотонального низкочастотного сигнала}
Для генерации гармонического сигнала воспользуемся формулой $s(t) = A*cos(2*\pi * f*t + \varphi)$, где А - амплитуда сигнала, f - частота, t - вектор отсчетов времени, $\varphi$ - смещение по фазе.

\subsection{Типы модуляции}

\subsubsection{Частотная модуляция}
При угловой модуляции в несущем гармоническом колебании $u(t) = U_m cos(\omega t + \varphi)$  значение амплитуды колебаний $U_m$ остается постоянным, а информация $s(t)$ переносится либо на частоту $\omega$, либо на фазовый угол $\varphi$. В обоих случаях текущее значение фазового угла гармонического колебания $u(t)$ определяет аргумент $\psi (t) = \omega t + \varphi$ , который называется полной фазой колебания.
Частотная модуляция выполняется по закону: 
\begin{equation}
	u(t) = U_m cos(\omega_0 t + k \int_{0}^{t} s(t) dt)
\end{equation}
Изображение сигнала после фазовой модуляции приведено ниже на риунке \ref{pic:Freq_mod_theor} :
\begin{figure}[H]
	\begin{center}
		\includegraphics[scale=0.7]{Freq_mod_theor}
		\caption{Фазовая модуляция сигнала} 
		\label{pic:Freq_mod_theor} % название для ссылок внутри кода
	\end{center}
\end{figure}
Данный вид модуляции внешне совпадает с фазовой, на практике необходимо иметь априорную информацию о типе модуляции при решении задачи демодуляции.

\subsubsection{Фазовая модуляция}
При фазовой модуляции (phase modulation – PM) значение фазового угла постоянной несущей частоты колебаний $\omega_0$ пропорционально амплитуде модулирующего сигнала $s(t)$. Соответственно, уравнение ФМ–сигнала определяется выражением: 
\begin{equation}
	u(t) = U_m cos(\omega_0 t + k s(t))
\end{equation}
Изображение сигнала после фазовой модуляции приведено ниже на риунке \ref{pic:Phase_mod_theor} :
\begin{figure}[H]
	\begin{center}
		\includegraphics[scale=0.7]{Phase_mod_theor}
		\caption{Фазовая модуляция сигнала} 
		\label{pic:Phase_mod_theor} % название для ссылок внутри кода
	\end{center}
\end{figure}

\section{Ход работы}
Код программы представлен ниже - два листинга, относящихся к фазовой модуляции -  \ref{code:code_1} и \ref{code:Phasemod}:
\lstinputlisting[
	label=code:code_1,
	caption={Код в МатЛаб},% для печати символ '_' требует выходной символ '\'
]{Code_1.m}
\lstinputlisting[
	label=code:Phasemod,
	caption={Код в МатЛаб},% для печати символ '_' требует выходной символ '\'
]{Phasemod.m}
В коде применены функции pmmod и pmdemod.

Далее приведен листинг кода частотной модуляции:
\lstinputlisting[
	label=code:Freqmod,
	caption={Код в МатЛаб},% для печати символ '_' требует выходной символ '\'
]{Freqmod.m}
В коде применены функции fmmod и fmdemod.

\subsection{Генерация однотонального сигнала}
Получим обычный гармонический сигнал. Он представлен на рисунке \ref{pic:signal_one_tone}:
\begin{figure}[H]
	\begin{center}
		\includegraphics[scale=0.7]{signal_one_tone}
		\caption{Гармонический сигнал $s(t) = A*cos(2*\pi * f*t + \varphi)$} 
		\label{pic:signal_one_tone} % название для ссылок внутри кода
	\end{center}
\end{figure}
Для однотонального сигнала спектр выглядит так:
\begin{figure}[H]
	\begin{center}
		\includegraphics[scale=0.7]{signal_one_tone_spec}
		\caption{Спектр гармонического сигнала $s(t) = A*cos(2*\pi * f*t + \varphi)$} 
		\label{pic:signal_one_tone_spec} % название для ссылок внутри кода
	\end{center}
\end{figure}

\subsection{Частотная модуляция}

\begin{figure}[H]
	\begin{center}
		\includegraphics[scale=0.7]{freq_mod_sig_1000_0}
		\caption{Частотно-модулированный сигнал} 
		\label{pic:freq_mod_sig_1000_0} % название для ссылок внутри кода
	\end{center}
\end{figure}
Видно, что максимум частоты выходного сигнала приходится на максимум амплитуды исходного сигнала.
\begin{figure}[H]
	\begin{center}
		\includegraphics[scale=0.7]{freq_mod_sig_spec_1000_0}
		\caption{Спектр частотно-модулированного сигнала} 
		\label{pic:freq_mod_sig_spec_1000_0} % название для ссылок внутри кода
	\end{center}
\end{figure}

\subsection{Фазовая модуляция}

\begin{figure}[H]
	\begin{center}
		\includegraphics[scale=0.7]{phase_mod_sig_carr}
		\caption{Фазово-модулированный сигнал (с отображением несущей гармоники)} 
		\label{pic:phase_mod_sig_carr} % название для ссылок внутри кода
	\end{center}
\end{figure}
Видно, что максимум частоты выходного сигнала приходится на максимум производной исходного сигнала.
\begin{figure}[H]
	\begin{center}
		\includegraphics[scale=0.7]{phase_mod_sig_carr_spec}
		\caption{Спектр фазово-модулированного сигнала} 
		\label{pic:phase_mod_sig_carr_spec} % название для ссылок внутри кода
	\end{center}
\end{figure}

\subsection{Демодуляция}
Произведем демодуляцию модулированных сигналов.
Демодуляция фазовой модуляции представлена на Рис.\ref{pic:phase_demod_sig} и \ref{pic:phase_demod_sig_spec}:
\begin{figure}[H]
	\begin{center}
		\includegraphics[scale=0.7]{phase_demod_sig}
		\caption{Фазово-демодулированный сигнал} 
		\label{pic:phase_demod_sig} % название для ссылок внутри кода
	\end{center}
\end{figure}
\begin{figure}[H]
	\begin{center}
		\includegraphics[scale=0.7]{phase_demod_sig_spec}
		\caption{Спектр фазово-демодулированного сигнала} 
		\label{pic:phase_demod_sig_spec} % название для ссылок внутри кода
	\end{center}
\end{figure}
Демодуляция частотной модуляции представлена на Рис.\ref{pic:freq_demod_sig} и \ref{pic:freq_demod_sig_spec}:
\begin{figure}[H]
	\begin{center}
		\includegraphics[scale=0.7]{freq_demod_sig}
		\caption{Частотно-демодулированный сигнал} 
		\label{pic:freq_demod_sig} % название для ссылок внутри кода
	\end{center}
\end{figure}
\begin{figure}[H]
	\begin{center}
		\includegraphics[scale=0.7]{freq_demod_sig_spec}
		\caption{Спектр частотно-демодулированного сигнала} 
		\label{pic:freq_demod_sig_spec} % название для ссылок внутри кода
	\end{center}
\end{figure}

Видно, что сигналы были демодулированы в обоих случаях, при чем с хорошей точностью, что говорит об эффективности использования таких методов модуляции и демодуляции.

\section{Выводы}

В данной работе нами были исследованы типы аналоговой модуляции и демодуляции, а именно - фазовая и частотная модуляции и демодуляции. Также были построены спектры этих сигналов. Далее была рповедена демодуляция сигнала. В обоих случах демодуляция правльно распознала исходный информационный сигнал.

Стоит отметить, что частотная модуляция применяется для высококачественной передачи звукового сигнала в радиовещании, для звукового сопровождения телевизионных программах.
\end{document}
