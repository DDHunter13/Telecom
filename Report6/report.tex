\include{settings}

\setcounter{tocdepth}{3}

\begin{document}	% начало документа

% Титульная страница
\include{titlepage}

% Содержание
\include{ToC}

 
\section{Цель работы}
Изучение различных методов модуляции цировых сигналов.
 
\section{Постановка задачи}
В ходе работы нам необходимо получить различные сигналы используя BPSK, PSK, OQPSK, genQAM, MSK, M-FSK модуляторы. В ходе работы с модуляторами необходимо построить их сигнальные созвездия. Затем провести сравнение изученных методов модуляции цифровых сигналов.



\section{Теоретическая информация}

\subsection{Типы цифровой модуляции}

Цифровая модуляция и демодуляция включают в себя две стадии. При модуляции цифровое сообщение сначала преобразуется в аналоговый модулирующий сигнал с помощью функции modmap, а затем осуществляется аналоговая модуляция. При демодуляции сначала получается аналоговый демодулированный сигнал, а затем он преобразуется в цифровое сообщение с помощью функции demodmap.
 
Аналоговый несущий сигнал модулируется цифровым битовым потоком.
Существуют три фундаментальных типа цифровой модуляции (или шифтинга) и один гибридный:
\begin{enumerate}
\item ASK – Amplitude shift keying (Амплитудная двоичная модуляция).
\item FSK – Frequency shift keying (Частотая двоичная модуляция).
\item PSK – Phase shift keying (Фазовая двоичная модуляция).
\item ASK/PSK.
\end{enumerate}
Одна из частных реализаций схемы ASK/PSK - QAM - Quadrature Amplitude Modulation (квадратурная амплитудная модуляция (КАМ). Это метод объединения двух AM-сигналов в одном канале. Он позваляет удвоить эффективную пропускную способность. В QAM используется две несущих с одинаковой частотой но с разницей в фазе на четверть периода.
Частотная модуляция представляет логическую единицу интервалом с большей частотой, чем ноль.
Фазовый сдвиг представляет «0» как сигнал без сдвига, а «1» как сигнал со сдвигом.
BPSK использует единственный сдвиг фазы между «0» и «1» — 180 градусов, половина периода.
QPSK использует 4 различных сдвига фазы (по четверти периода) и может кодировать 2 бита в символе (01, 11, 00, 10).

\subsubsection{BPSK, PSK}
BPSK и PSK - модуляция со сдвиглм фазы сигнала без изменения амплитуды. В PSK их может быть множество, в BPSK - один (на $\pi$).

Изображения сигнального созвездия и схемы модулятора BPSK приведены ниже на следующих рисунках:
\begin{figure}[H]
	\begin{center}
		\includegraphics[width=1\linewidth]{BPSK_Scheme_theor.png}
		\caption{Схема устройства модулятора BPSK.} %% подпись к рисунку
		\label{BPSK_Scheme_theor} %% метка рисунка для ссылки на него
	\end{center}
\end{figure}
\begin{figure}[H]
	\begin{center}
		\includegraphics[width=0.5\linewidth]{BPSK_Sig_Con_theor.png}
		\caption{Сигнальное созвездие BPSK.} %% подпись к рисунку
		\label{BPSK_Sig_Con_theor} %% метка рисунка для ссылки на него
	\end{center}
\end{figure}

\subsubsection{genQAM, OQPSK}
При квадратурной амплитудной модуляции (КАМ) изменяется как фаза, так и амплитуда несущего сигнала. Это позволяет увеличить количество кодируемых в единицу времени бит и при этом повысить помехоустойчивость их передачи по каналу связи. В настоящее время число кодируемых информационных бит на одном интервале может достигать 8-9, а число состояний сигнала в сигнальном пространстве, соответственно – 256…512.
Квадратурное представление сигнала заключается в выражении колебания линейной комбинацией двух ортогональных составляющих – квадратурной и синфазной:
\begin{equation}
	S(t) = x(t) sin(\omega t + \varphi) cos(\omega t + \varphi)
\end{equation}
где x(t) и y(t) – биполярные дискретные сигналы.

Четырехфазная ФМ со сдвигом (OQPSK – Offset QPSK) позволяет избежать скачков фазы на 180° и, следовательно, глубокой модуляции огибающей. Формирование сигнала в модуляторе OQPSK происходит так же, как и в модуляторе ФМ-4, за исключением того, что манипуляционные элементы информационных последовательностей $x(t)$ и $y(t)$ смещены во времени на длительность одного элемента $Т$, (Рис.\ref{manip_sig_theor}). Изменение фазы при таком смещении модулирующих потоков определяется лишь одним элементом последовательности, а не двумя, как при ФМ 4. В результате скачки фазы на 180° отсутствуют, так как каждый элемент последовательности, поступающий на вход модулятора синфазного или квадратурного канала, может вызвать изменение фазы на $0$, $+90°$ или $-90°$.
\begin{figure}[H]
	\begin{center}
		\includegraphics[width=1\linewidth]{manip_sig_theor.png}
		\caption{Формирование манипулирующих сигналов} %% подпись к рисунку
		\label{manip_sig_theor} %% метка рисунка для ссылки на него
	\end{center}
\end{figure} 
Преобразованные таким образом сигналы передаются в одном канале. Поскольку один и тот же физический канал используется для передачи двух сигналов, то скорость передачи КАМ-сигнала в отличие от АМ-сигнала в два раза выше.

Ниже показана структурная схема модулятора и диаграмма состояний (сигнальное созвездие) системы КАМ-16, в которой $x(t)$ и $y(t)$ принимают значения $\pm 1, \pm 3$ (4-х уровневая КАМ). 
\begin{figure}[H]
	\begin{center}
		\includegraphics[width=1\linewidth]{QAM_16_theor.png}
		\caption{Модуляция КАМ-16 и ее сигнальное созвездие} %% подпись к рисунку
		\label{QAM_16_theor} %% метка рисунка для ссылки на него
	\end{center}
\end{figure} 

\subsubsection{MSK}
Частотная манипуляция с минимальным сдвигом (англ. Minimal Shift Keying (MSK)) представляет собой способ модуляции, при котором не происходит скачков фазы и изменение частоты происходит в моменты пересечения несущей нулевого уровня. MSK характеризуется тем, что значение частот соответствующих логическим «0» и «1» отличаются на величину равную половине скорости передачи данных. Другими словами, индекс модуляции равен 0,5.

Изображения сигнального созвездия и схемы модулятора MSK приведены ниже на риунках:
\begin{figure}[H]
	\begin{center}
		\includegraphics[width=1\linewidth]{MSK_scheme_theor.png}
		\caption{Структурная схема формирования MSK на основе FM модулятора.} %% подпись к рисунку
		\label{MSK_scheme_theor} %% метка рисунка для ссылки на него
	\end{center}
\end{figure}
\begin{figure}[H]
	\begin{center}
		\includegraphics[width=1\linewidth]{MSK_diag_theor.png}
		\caption{Полная фазовая диаграмма при MSK для 4-х бит информации.} %% подпись к рисунку
		\label{MSK_diag_theor} %% метка рисунка для ссылки на него
	\end{center}
\end{figure}
\begin{figure}[H]
	\begin{center}
		\includegraphics[width=0.5\linewidth]{MSK_sig_con_theor.png}
		\caption{Сигнальное созвездие MSK.} %% подпись к рисунку
		\label{MSK_sig_con_theor} %% метка рисунка для ссылки на него
	\end{center}
\end{figure}

\subsubsection{MFSK}
Можно построить и модулятор многопозиционной частотной модуляции. В этом случае будет использовано большее количество синусоидальных генераторов, а для управления коммутатором потребуется многоразрядное двоичное число.

Сигналы в многопозиционной частотной модуляции могут быть описаны в соответствии со следующим выражением:
\begin{equation}
	s_1 (t) =  cos(\omega_1 t); s_2 (t) =  cos(\omega_2 t); ...; s_N (t) =  cos(\omega_N t); 
\end{equation}
формула сигнала 1 многопозиционной частотной модуляции,  формула сигнала 2 многопозиционной частотной модуляции, …,  формула сигнала N многопозиционной частотной модуляции (3)
где $s_1$ используется для передачи первого состояния символа;
$s_2$ — для передачи второго состояния символа;
$s_N$ — для передачи N-го состояния символа.

Использование многопозиционной частотной модуляции позволяет реализовать высокочастотный сигнал с постоянной амплитудой. Такой сигнал позволяет строить радиопередатчики с максимальным кпд, так как при применении сигнала с постоянной амплитудой, усилитель мощности радиопередатчика работает в оптимальном режиме.

\section{Ход работы}

Реализация различных типов модуляций с помощью MATLAB:

\lstinputlisting[
	label=code:DiscrMod2,
	caption={Код в МатЛаб},% для печати символ '_' требует выходной символ '\'
]{DiscrMod2.m}

Результаты выполнения представлены на рисунках ниже:

\subsection{BPSK-модуляция}
 Код для получение BPSK модуляции расположен в строках 1-17 на \ref{code:DiscrMod}:
\begin{figure}[H]
	\begin{center}
		\includegraphics[width=0.5\linewidth]{Inp_msg_BPSK.png}
		\caption{Входной сигнал BPSK.} %% подпись к рисунку
		\label{Inp_msg_BPSK} %% метка рисунка для ссылки на него
	\end{center}
\end{figure}

\begin{figure}[H]
	\begin{center}
		\includegraphics[width=0.5\linewidth]{sig_con_bpsk.png}
		\caption{Сигнальное созвездие BPSK.} %% подпись к рисунку
		\label{sig_con_bpsk} %% метка рисунка для ссылки на него
	\end{center}
\end{figure}

\begin{figure}[H]
	\begin{center}
		\includegraphics[width=0.5\linewidth]{Inp_msg_BPSK.png}
		\caption{Демодулированный сигнал BPSK.} %% подпись к рисунку
		\label{Inp_msg_BPSK} %% метка рисунка для ссылки на него
	\end{center}
\end{figure}
Видно, что демодулированный сигнал совпал с исходным.

\subsection{PSK-модуляция}
 Код для получение BPSK модуляции расположен в строках 20-36 на \ref{code:DiscrMod}:
\begin{figure}[H]
	\begin{center}
		\includegraphics[width=0.5\linewidth]{Inp_msg_PSK.png}
		\caption{Входной сигнал PSK.} %% подпись к рисунку
		\label{Inp_msg_PSK} %% метка рисунка для ссылки на него
	\end{center}
\end{figure}
 
\begin{figure}[H]
	\begin{center}
		\includegraphics[width=0.5\linewidth]{sig_con_psk.png}
		\caption{Сигнальное созвездие PSK.}
		\label{sig_con_psk}
	\end{center}
\end{figure}

\begin{figure}[H]
	\begin{center}
		\includegraphics[width=0.5\linewidth]{Demod_msg_PSK.png}
		\caption{Демодулированный сигнал PSK.} %% подпись к рисунку
		\label{Demod_msg_PSK} %% метка рисунка для ссылки на него
	\end{center}
\end{figure}
Видно, что демодулированный сигнал совпал с исходным.
\subsection{OQPSK-модуляция}
 Код для получение BPSK модуляции расположен в строках 38-54 на \ref{code:DiscrMod}:
\begin{figure}[H]
	\begin{center}
		\includegraphics[width=0.5\linewidth]{Inp_msg_OQPSK.png}
		\caption{Входной сигнал OQPSK.} %% подпись к рисунку
		\label{Inp_msg_OQPSK} %% метка рисунка для ссылки на него
	\end{center}
\end{figure}

\begin{figure}[H]
	\begin{center}
		\includegraphics[width=0.5\linewidth]{sig_con_oqpsk.png}
		\caption{Сигнальное созвездие OQPSK.} %% подпись к рисунку
		\label{sig_con_oqpsk} %% метка рисунка для ссылки на него
	\end{center}
\end{figure}

\begin{figure}[H]
	\begin{center}
		\includegraphics[width=0.5\linewidth]{Demod_msg_OQPSK.png}
		\caption{Демодулированный сигнал OQPSK.} %% подпись к рисунку
		\label{Demod_msg_OQPSK} %% метка рисунка для ссылки на него
	\end{center}
\end{figure}
Видно, что демодулированный сигнал совпал с исходным.
\subsection{genQAM-модуляция}
 Код для получение BPSK модуляции расположен в строках 57-73 на \ref{code:DiscrMod}:
\begin{figure}[H]
	\begin{center}
		\includegraphics[width=0.5\linewidth]{Inp_msg_genQAM.png}
		\caption{Входной сигнал genQAM.} %% подпись к рисунку
		\label{Inp_msg_genQAM} %% метка рисунка для ссылки на него
	\end{center}
\end{figure}

\begin{figure}[H]
	\begin{center}
		\includegraphics[width=0.5\linewidth]{sig_con_genqam.png}
		\caption{Сигнальное созвездие genQAM.}
		\label{sig_con_genqam}
	\end{center}
\end{figure}

\begin{figure}[H]
	\begin{center}
		\includegraphics[width=0.5\linewidth]{Demod_msg_genQAM.png}
		\caption{Демодулированный сигнал genQAM.} %% подпись к рисунку
		\label{Demod_msg_genQAM} %% метка рисунка для ссылки на него
	\end{center}
\end{figure}
Видно, что демодулированный сигнал совпал с исходным.
\subsection{MSK-модуляция}
 Код для получение BPSK модуляции расположен в строках 76-92 на \ref{code:DiscrMod}:
\begin{figure}[H]
	\begin{center}
		\includegraphics[width=0.5\linewidth]{Inp_msg_MSK.png}
		\caption{Входной сигнал MSK.} %% подпись к рисунку
		\label{Inp_msg_MSK} %% метка рисунка для ссылки на него
	\end{center}
\end{figure}

\begin{figure}[H]
	\begin{center}
		\includegraphics[width=0.5\linewidth]{sig_con_msk.png}
		\caption{Сигнальное созвездие MSK.} %% подпись к рисунку
		\label{sig_con_msk} %% метка рисунка для ссылки на него
	\end{center}
\end{figure}

\begin{figure}[H]
	\begin{center}
		\includegraphics[width=0.5\linewidth]{Demod_msg_MSK.png}
		\caption{Демодулированный сигнал MSK.} %% подпись к рисунку
		\label{Demod_msg_MSK} %% метка рисунка для ссылки на него
	\end{center}
\end{figure}
Как можно видеть, при использовании MSK выходной сигнал имеет задержку при демодуляции.
Видно, что демодулированный сигнал совпал с исходным.
\subsection{MFSK-модуляция}
В Simulink была построена модель MFSK-модулятора, результаты работы совпали с ожидаемыми, входная последовательность совпала с выходной.
\begin{figure}[H]
	\begin{center}
		\includegraphics[width=1\linewidth]{MFSK_Mod_theor.png}
		\caption{Simulink-модель MFSK.} %% подпись к рисунку
		\label{MFSK_Mod_theor} %% метка рисунка для ссылки на него
	\end{center}
\end{figure}

\begin{figure}[H]
	\begin{center}
		\includegraphics[width=1.1\linewidth]{MFSK_Gen.png}
		\caption{Графики входного сигнала, задержанного сигнала, модулированного сигнала, сигнала ошибки с задержанным сигналом, выходного сигнала MFSK.} %% подпись к рисунку
		\label{MFSK_Gen} %% метка рисунка для ссылки на него
	\end{center}
\end{figure}
  
 
\section{Выводы}

Квадратурная амплитудная манипуляция (QAM) — манипуляция, при которой изменяется как фаза, так и амплитуда сигнала, что позволяет увеличить количество информации, передаваемой одним состоянием сигнала. 

Фазовая манипуляция (PSK) — модуляция, при которой фаза несущего колебания меняется скачкообразн. 

При квадратурной фазовой манипуляции (QPSK) используется созвездие из четырёх точек, размещённых на равных расстояниях на окружности. Имеется 4 фазовых смещений, при этом в QPSK на символ приходится два бита. 

Частотная манипуляция с минимальным сдвигом (MSK) представляет собой способ модуляции, при котором не происходит скачков фазы и изменение частоты происходит в моменты пересечения несущей нулевого уровня. Принцип MSK таков, что значение частот соответствующих логическим «0» и «1» отличаются на величину равную половине скорости передачи данных.

Уровень модуляции определяет количество состояний несущей, используемых для передачи информации. Чем выше этот уровень, тем большими скоростными возможностями и меньшей помехоустойчивостью обладает модуляция. Число бит, передаваемых одним состоянием, определяется как $Log (N)$, где N — уровень модуляции.
\end{document}