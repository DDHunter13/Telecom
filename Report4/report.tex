\include{settings}

\setcounter{tocdepth}{3}

\begin{document}	% начало документа

% Титульная страница
\include{titlepage}

% Содержание
\include{ToC}


\section{Цель работы}
Изучение амплитудной модуляции и демодуляции сигнала.

\section{Постановка задачи}
Необходимо сгенерировать однотональныйнизкочастотный сигнал, выполнить амплитудную модуляцию этого сигнала, затем модуляцию с подавлением несущей, однополосную модуляцию. Далее для всех типов модуляции осуществить синхронное детектирование. Рассмотреть спектры сигналов после модуляции и после детектирования. Расчитать КПД модуляции.

\section{Теоретическая информация}

\subsection{Модуляция}
Перенос спектра сигналов из низкочастотной области на заданную частоту, т.е. в выделенную для их передачи область высоких частот выполняется операцией, которая называется модуляция. Пусть исходный низкочастотный сигнал - $s(t)$ .

В канале связи для передачи данного сигнала формируется вспомогательный периодический высокочастотный сигнал $u(t)   =   f(t;   a_1,   a_2,   ...   a_m)$. Параметры $a_i$ определяют форму сигнала. Значения этих параметров в отсутствие модуляции являются величинами постоянными. Если на один из этих параметров перенести сигнал $s(t)$, т.е. сделать его значение пропорционально зависимым от значения $s(t)$ во времени, то форма сигнала $u(t)$ приобретает новое свойство. Она служит для переноса информации, содержащейся в сигнале $s(t)$. Сигнал $u(t)$ в таком случае называется несущим сигналом, или несущей,  а физический процесс переноса информации на
параметры несущего сигнала – модуляцией. Обратную операцию выделения модулирующего сигнала из модулированного колебания называют демодуляцией или детектированием.

\subsection{Генерация однотонального низкочастотного сигнала $s(t)$}
Для генерации гармонического сигнала воспользуемся формулой $s(t) = A*cos(2*\pi * f*t + \varphi)$, где А - амплитуда сигнала, f - частота, t - вектор отсчетов времени, $\varphi$ - смещение по фазе.

\subsection{Типы модуляции}
\subsubsection{Амплитудная модуляция}
Формула АМ имеет вид: 
\begin{equation}
	u(t) = (1 + M U_m cos(\Omega t)) cos(\omega_0 t + \varphi _0)
\end{equation}
Спектр амплитудно-модулированного сигнала представлен на Рис.\ref{pic:spec_an_mod_theor}:
\begin{figure}[H]
	\begin{center}
		\includegraphics[scale=0.7]{spec_an_mod_theor}
		\caption{Спектр амплитудно-модулированного сигнала} 
		\label{pic:spec_an_mod_theor} % название для ссылок внутри кода
	\end{center}
\end{figure}
В настоящее время такая модуляция применяется крайне редко из-за низкого КПД.

\subsubsection{Амплитудная модуляция с подавлением несущей}
Основная доля мощности АМ – сигнала приходится на несущую частоту. При АМ с подавлением несущей производится перемножение двух сигналов – модулирующего и несущего, для подавления несущего колебания, соответственно, КПД модуляции становится 100\%. Формула такой модуляции:
\begin{equation}
	u(t) = M U_m cos(\Omega t) cos(\omega_0 t + \varphi _0)
\end{equation}
Спектр балансно-модулированного сигнала представлен на Рис.\ref{pic:spec_an_mod_carrier_theor}:
\begin{figure}[H]
	\begin{center}
		\includegraphics[scale=0.7]{spec_an_mod_carrier_theor}
		\caption{Спектр балансно-модулированного сигнала} 
		\label{pic:spec_an_mod_carrier_theor} % название для ссылок внутри кода
	\end{center}
\end{figure}

\subsubsection{Однополосная модуляция}
При идентичности информации в группах верхних и нижних боковых частот нет необходимости в их одновременной передаче. Одна из них перед подачей сигнала в канал связи может быть удалена, чем достигается двукратное сокращение полосы занимаемых сигналом частот. Уравнение сигнала с одной боковой полосой (ОБП) приведено ниже:
 \begin{equation}
	u(t) = U_m cos(\Omega t) cos(\omega_0 t + \varphi _0) + \frac{U_m}{2} \sum_{n=1}^{N}  M_n cos((\omega_0 + \Omega_n) t + \varphi _0 + \Phi _n)
\end{equation}
Внешняя форма ОБП – сигнала сходна с обычным АМ – сигналом, но ее огибающая имеет меньшую амплитуду по сравнению с АМ. Для демодуляции ОБП – сигнала может использоваться как двухполупериодное, так и синхронное детектирование, со всеми особенностями, присущими этим методам. Результаты демодуляции отличаются от демодуляции АМ – сигналов только меньшей амплитудой выходных сигналов.

Спектр однополосно-модулированного сигнала и структурная схема соответствующего устройства представлены на Рис.\ref{pic:spec_an_mod_singleband_theor}:
\begin{figure}[H]
	\begin{center}
		\includegraphics[scale=0.7]{spec_an_mod_singleband_theor}
		\caption{Спектр однополосно-модулированного сигнала} 
		\label{pic:spec_an_mod_singleband_theor} % название для ссылок внутри кода
	\end{center}
\end{figure}

\subsubsection{Демодуляция с помощью синхронного детектирования}
При синхронном детектировании модулированный сигнал умножается на опорное колебание с частотой несущего колебания:
 \begin{equation}
	y(t) = U(t) cos(\omega_0 t) cos(\omega_0 t) = \frac{U(t)}{2} (1 + cos(2\omega_0 t))
\end{equation}
Сигнал разделяется на два слагаемых, первое из которых повторяет исходный модулирующий сигнал, а второе повторяет модулированный сигнал на удвоенной несущей частоте 2$\omega_0$. Форма новой несущей при синхронном детектировании является чистой гармоникой, в отличие от двухполупериодного детектирования, где новая несущая содержит дополнительные гармоники более высоких частот. 

Физический амплитудный спектр сигналов после демодуляции подобен спектру двухполупериодного детектирования, но однозначно соотносится со спектром входного модулированного сигнала: амплитуды гармоник модулированного сигнала на частоте 2$\omega_0$ в два раза меньше амплитуд входного сигнала, постоянная составляющая равна амплитуде несущей частоты $\omega_0$ и не зависит от глубины модуляции, амплитуда информационного демодулированного сигнала в два раза меньше амплитуды исходного модулирующего сигнала. 

Особенностью синхронного детектирования является независимость от глубины модуляции, т.е. коэффициент модуляции сигнала может быть больше единицы. При синхронном детектировании требуется точное совпадение фаз и частот опорного колебания демодулятора и несущей гармоники АМ-сигнала.

\subsubsection{КПД модуляции}
КПД амплитудной модуляции зависит от коэффициента модуляции и может быть расчитано по следующей формуле:
 \begin{equation}
	\eta (t) =\frac{ U_m^2(t) M^2}{4 P_U}  = \frac{M^2}{2 + M^2} 
\end{equation}



\section{Ход работы}
Код программы представлен ниже \ref{code:code_1}:
\lstinputlisting[
	label=code:code_1,
	caption={Код в МатЛаб},% для печати символ '_' требует выходной символ '\'
]{Code_1.m}
В коде применены функции ammod и ssbmod.

\subsection{Генерация однотонального сигнала}
Для начала получим обычный гармонический сигнал. Сгенерированный сигнал представлен на рисунке \ref{pic:signal_one_tone}:
\begin{figure}[H]
	\begin{center}
		\includegraphics[scale=0.7]{signal_one_tone}
		\caption{Гармонический сигнал $s(t) = A*cos(2*\pi * f*t + \varphi)$} 
		\label{pic:signal_one_tone} % название для ссылок внутри кода
	\end{center}
\end{figure}
Для однотонального сигнала спектр выглядит следующим образом:
\begin{figure}[H]
	\begin{center}
		\includegraphics[scale=0.7]{signal_one_tone_spec}
		\caption{Спектр гармонического сигнала $s(t) = A*cos(2*\pi * f*t + \varphi)$} 
		\label{pic:signal_one_tone_spec} % название для ссылок внутри кода
	\end{center}
\end{figure}

\subsection{Амплитудная модуляция}
Сгенерированный однотональный сигнал подвергли амплитудной модуляции (при соотношении амплитуд инф./несущ. = 0.5). Сигнал после модуляции и его спектр представлены на рисунках \ref{pic:signal_modulated_0_5} и \ref{pic:mod_sig_spec_0_5}:
\begin{figure}[H]
	\begin{center}
		\includegraphics[scale=0.7]{signal_modulated_0_5}
		\caption{Амплитудно-модулированный сигнал ($M = 0.5$)} 
		\label{pic:signal_modulated_0_5} % название для ссылок внутри кода
	\end{center}
\end{figure}

\begin{figure}[H]
	\begin{center}
		\includegraphics[scale=0.7]{mod_sig_spec_0_5}
		\caption{Спектр амплитудно-модулированного сигнала ($M = 0.5$)} 
		\label{pic:mod_sig_spec_0_5} % название для ссылок внутри кода
	\end{center}
\end{figure}
Спектр содержит гармонику информационного сигнала и две гармоники несущего сигнала по бокам.

Теперь будем изменять амплитуду модулирующего (информационного) сигнала для наблюдения изменения сигнала с модуляцией (его коэффициента модуляции M).

Пусть $M = 0.2$.
\begin{figure}[H]
	\begin{center}
		\includegraphics[scale=0.7]{signal_modulated_0_2}
		\caption{Амплитудно-модулированный сигнал ($M = 0.2$)} 
		\label{pic:signal_modulated_0_2} % название для ссылок внутри кода
	\end{center}
\end{figure}
\begin{figure}[H]
	\begin{center}
		\includegraphics[scale=0.7]{mod_sig_spec_0_2}
		\caption{Спектр амплитудно-модулированного сигнала ($M = 0.2$)} 
		\label{pic:mod_sig_spec_0_2} % название для ссылок внутри кода
	\end{center}
\end{figure}

Пусть $M = 1$.
\begin{figure}[H]
	\begin{center}
		\includegraphics[scale=0.7]{signal_modulated_1_0}
		\caption{Амплитудно-модулированный сигнал ($M = 1$)} 
		\label{pic:signal_modulated_1_0} % название для ссылок внутри кода
	\end{center}
\end{figure}
\begin{figure}[H]
	\begin{center}
		\includegraphics[scale=0.7]{mod_sig_spec_1_0}
		\caption{Спектр амплитудно-модулированного сигнала ($M = 1$)} 
		\label{pic:mod_sig_spec_1_0} % название для ссылок внутри кода
	\end{center}
\end{figure}

Пусть $M = 2$.
\begin{figure}[H]
	\begin{center}
		\includegraphics[scale=0.7]{signal_modulated_2_0}
		\caption{Амплитудно-модулированный сигнал ($M = 2$)} 
		\label{pic:signal_modulated_2_0} % название для ссылок внутри кода
	\end{center}
\end{figure}
\begin{figure}[H]
	\begin{center}
		\includegraphics[scale=0.7]{mod_sig_spec_2_0}
		\caption{Спектр амплитудно-модулированного сигнала ($M = 2$)} 
		\label{pic:mod_sig_spec_2_0} % название для ссылок внутри кода
	\end{center}
\end{figure}

При M > 1 имеем случай перемодуляции, при M = 1 - случай глубокой модуляции, а при M < 1 - обычный случай модуляции без совмещений полупериодов гармонического сигнала огибающей.

\subsection{Амплитудная модуляция с подавлением несущей}
Подавление несущей осуществляется узкополосной фильтрацией сигнала на частоте информационного. Сигнал с АМ-ПН представлен на рисунке \ref{pic:signal_mod_carrier}:
\begin{figure}[H]
	\begin{center}
		\includegraphics[scale=0.7]{signal_mod_carrier}
		\caption{Сигнал с АМ-ПН} 
		\label{pic:signal_mod_carrier} % название для ссылок внутри кода
	\end{center}
\end{figure}
Подавление несущей приводит к тому, что основная мощность сигнала (приходящаяся на несущую гармонику) фильтруется, но такой сигнал не демодулируется. Решить такую проблему можно частичной фильтрацией несущей, то есть сохранение амплитуды этой гармоники ненулевой, но более низкой, чем у информационной составляющей.

\subsection{Однополосная амплитудная модуляция}
Помимо подавления несущей, можно избавиться от лишней (дублирующейся) боковой полосы спектра с помощью ФНЧ. Сигнал представлен на рисунке\ref{pic:signal_mod_singleband}:
\begin{figure}[H]
	\begin{center}
		\includegraphics[scale=0.7]{signal_mod_singleband}
		\caption{Сигнал с АМ-ОП} 
		\label{pic:signal_mod_singleband} % название для ссылок внутри кода
	\end{center}
\end{figure}

\subsection{Спектры АМ-ПН и АМ-ОП}
Ниже, на рисунке \ref{pic:signal_mod_carrier_sinband_specs}, приведены спектры сигналов после АМ-ПН и АМ-ОП.
\begin{figure}[H]
	\begin{center}
		\includegraphics[scale=0.7]{signal_mod_carrier_sinband_specs}
		\caption{Спектры сигнала с АМ-ПН и АМ-ОП} 
		\label{pic:signal_mod_carrier_sinband_specs} % название для ссылок внутри кода
	\end{center}
\end{figure}
На первом рисунке видны две полосы (без несущей), что соответствует АМ-ПН. Ниже приведён спектр, содержащий одну полосу, что соответствует АМ-ОП.

\subsection{Демодуляция с помощью синхронного детектирования}
Произведем демодуляцию модулированных сигналов с разными коэффициентами модуляции.

Пусть $M = 0.2$.
\begin{figure}[H]
	\begin{center}
		\includegraphics[scale=0.7]{sig_demod_0_2}
		\caption{Демодулированный сигнал ($M = 0.2$)} 
		\label{pic:sig_demod_0_2} % название для ссылок внутри кода
	\end{center}
\end{figure}

Пусть $M = 1$.
\begin{figure}[H]
	\begin{center}
		\includegraphics[scale=0.7]{sig_demod_1_0}
		\caption{Амплитудно-модулированный сигнал ($M = 1$)} 
		\label{pic:sig_demod_1_0} % название для ссылок внутри кода
	\end{center}
\end{figure}

Пусть $M = 2$.
\begin{figure}[H]
	\begin{center}
		\includegraphics[scale=0.7]{sig_demod_2_0}
		\caption{Амплитудно-модулированный сигнал ($M = 2$)} 
		\label{pic:sig_demod_2_0} % название для ссылок внутри кода
	\end{center}
\end{figure}

Как можно видеть, нелинейные искажения сигнала при демодуляции тем незначительнее, чем больше коэффициент модуляции.
Ниже приведен спектр демодулированного сигнала при M = 2.
\begin{figure}[H]
	\begin{center}
		\includegraphics[scale=0.7]{sig_demod_spec_2_0}
		\caption{Спектр демодулированного сигнала ($M = 2$)} 
		\label{pic:sig_demod_spec_2_0} % название для ссылок внутри кода
	\end{center}
\end{figure}
В сигнале появились низкочастотная составляющая и высокочастотные искажения, однако при применении полосового фильтра можно выделить искомый сигнал с достаточной точностью совпадающий с исходным.

При M = 5 имеем:
\begin{figure}[H]
	\begin{center}
		\includegraphics[scale=0.7]{sig_demod_spec_5_0}
		\caption{Спектр демодулированного сигнала ($M = 5$)} 
		\label{pic:sig_demod_spec_5_0} % название для ссылок внутри кода
	\end{center}
\end{figure} 
Низкочастотная составляющая значительно меньше по амплитуде, чем информационная, высокочастотные искажения так же стали более незначительны, чем при M = 2.

\subsection{КПД модуляции}
Ниже, на рисунке \ref{pic:Kpd_ampmod}, приведена зависимость КПД модуляции от амплитуды модулирующего сигнала (т.е. от коэффициента модуляции).
\begin{figure}[H]
	\begin{center}
		\includegraphics[scale=0.7]{Kpd_ampmod}
		\caption{Зависимость КПД модуляции от амплитуды модулирующего сигнала} 
		\label{pic:Kpd_ampmod} % название для ссылок внутри кода
	\end{center}
\end{figure}

\section{Выводы}

В ходе этой работы нами были исследованы типы аналоговой модуляции - амплитудная, с подавлением несущей и однополосная, также исследован способ демодуляции с помощью синхронного детектирования и определена зависимость КПД модуляции от коэффициента модуляции. Также были построены спектры модулированных сигналов, их вид совпал с ожидаемым результатом для каждого типа модуляции.

Основной результат, полученный в ходе работы - нами были получены представления о принципах аналоговой амплитудной модуляции. Она находит широкое применение: в системах телевизионного вещания, в системах звукового радиовещания и радиосвязи на длинных и средних волнах, в системе трехпрограммного проводного вещания.

\end{document}
