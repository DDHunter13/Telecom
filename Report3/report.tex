\include{settings}

\begin{document}	% начало документа

% Титульная страница
\include{titlepage}

% Содержание
\include{ToC}


\section{Цель работы}
Изучить воздействие фильра нижних частот на тестовый синал с шумом.

\section{Постановка задачи}
Сгенерировать тестовый гармонический сигнал с шумом, синтезировать ФНЧ, отфильтровать сигнал с шумом. Посмотреть, как ФНЧ влияет на спектр сигнала.

\section{Теоретическая информация}
\subsection{Генерация гармонического сигнала с шумом}
Для генерации обычного гармонического сигнала можно воспользоваться простейшей формулой $signal = A*cos(2*\pi * f*t + \varphi)$, где А - амплитуда сигнала, f - частота, t - вектор отсчетов времени, $\varphi$ - смещение по фазе.

Для добавления шума в исходный сигнал необходимо сложить его с другим сигналом, полученным по аналогичной формуле, но для другой частоты.

\subsection{Фильтр нижних частот}
Любой фильтр работает по принципу домножения сигнала в частотной области на некий коэффициент, зависящий от частоты. Таким образом, фильтр может ослаблять сигналы на одной частоте, и  оставлять неизменными, и даже усиливать, на другой. Так, фильтр нижних частот ослабляет частоты более высокие заданной границы, умножая их на маленький коэффициент. АЧХ такого фильтра представлена на Рис.\ref{pic:ach_fnc}:
\begin{figure}[H]
	\begin{center}
		\includegraphics[scale=0.7]{ach_fnc}
		\caption{АЧХ фильра нижних частот} 
		\label{pic:ach_fnc} % название для ссылок внутри кода
	\end{center}
\end{figure}

Для приближения реальной АЧХ к желаемой, используется аппроксимация. Наиболее используемые аппроксимации - Баттерворта, Чебышева. В данной работе будем использовать фильтр Баттерворта. АЧХ такого фильтра n-ого порадка можно вычислить по формуле:
\begin{equation}
	G^2(\omega) = \frac{G^2_0}{1 + \big( \frac{\omega}{\omega _c} \big) ^{2n}}
\end{equation}
где n - порядок фильтра, $\omega _c$ - частота среза, $G_0$ - коэффициент усиления на нулевой частоте.

\section{Ход работы}
Общий код программы представлен ниже \ref{code:code}:
\lstinputlisting[
	label=code:code,
	caption={Код в МатЛаб},% для печати символ '_' требует выходной символ '\'
]{Code.m}
\parindent=1cm
Его можно условно разделить на 2 части - генерация сигнала и выведение его графика и его спектра на экран, и фильтрация этого сигнала, с последующим выведением тех же графиков

\subsection{Генерация гармонического сигнала с шумом}
Для начала получим обычный гармонический сигнал. Пусть его частота будет 20 Гц. Сгенерированный сигнал представлен на рисунке \ref{pic:signal1}:
\begin{figure}[H]
	\begin{center}
		\includegraphics[scale=0.7]{signal1}
		\caption{Гармонический сигнал} 
		\label{pic:signal1} % название для ссылок внутри кода
	\end{center}
\end{figure}
На графике видимо обычную синусоиду.

Затем сгенерироем еще одну синусоиду с другой, более высокой частотой, прибавив его к уже полеченной гармонике. Результат внесния шума в сигнал виден на рисунке \ref{pic:signal2}:
\begin{figure}[H]
	\begin{center}
		\includegraphics[scale=0.7]{signal2}
		\caption{Гармонический сигнал с шумом} 
		\label{pic:signal2} % название для ссылок внутри кода
	\end{center}
\end{figure}

Далее получим спектр сигнала с помощью преобразования Фурье, встроенного в МатЛаб. Спектр гармонического сигнала с шумом приведен на рисунке \ref{pic:signal2_fft}:
\begin{figure}[H]
	\begin{center}
		\includegraphics[scale=0.7]{signal2_fft}
		\caption{Спектр зашумленной гармоники} 
		\label{pic:signal2_fft} % название для ссылок внутри кода
	\end{center}
\end{figure}
Видно, что в сигнале присутствуют 2 гармоники разной частоты.

\subsection{Фильтрация сигнала}
Для фильтрации будем использовать ФНЧ Баттерворта 4-ого порядка. Коэффициенты фильтра получим с помощью встроенной в МатЛаб функции butter. В качестве аргумента указываем порядок фильтра и величину $Fn*2/Fd$, где Fn - частота полезного сигнала, а Fd - частота дискретизации.
Полученные коэффициенты задаются как аргументы в функцию фильтрации filter. На выходе имеем отфильтрованый сигнал. Его можно увидеть на рисунке \ref{pic:filter_signal}:
\begin{figure}[H]
	\begin{center}
		\includegraphics[scale=0.7]{filter_signal}
		\caption{Сигнал после прохождения фильтра} 
		\label{pic:filter_signal} % название для ссылок внутри кода
	\end{center}
\end{figure}
Видим отфильтрованый сигнал. Максимальная амплитуда немного уменьшена из-за коэффициента ослабления филтра, и сигнал устанавливается с небольшой задержкой.

Спектр данного сигнала, полученный также с помощью преобразования Фурье, приведен на рис. \ref{pic:filter_signal_fft}:
\begin{figure}[H]
	\begin{center}
		\includegraphics[scale=0.7]{filter_signal_fft}
		\caption{Спектр отфильтрованного сигнала} 
		\label{pic:filter_signal_fft} % название для ссылок внутри кода
	\end{center}
\end{figure}
На рисунке видна одна гармоника, т.е. фильтр верно отсек гармонику шума, внесенного нами в сигнал.

\section{Выводы}

Нами исследовано прохождение сигнала через линейную цепь фильтра нижних частот. На примере зашемленного гармонического сигнала удалось получить представление о том, что происходит с сигналом при фильтрации. А именно - частотная характеристика сигнала проходит свертку с окном желаемой АЧХ. Сложность состоит в том, что получить в качестве окна идеальный прямоугольник невозможно. Поэтому используются различные метода аппроксимации идеальной АЧХ фильтра. Неидеальностью АЧХ фильтра можно объяснипть неполное подавление шума, особенно на частотах, близких к частоте среза, т.к. аппроксимация иммеет неидеальный наклон кривой после частоты среза.

\end{document}
